\documentclass[]{elsarticle} %review=doublespace preprint=single 5p=2 column
%%% Begin My package additions %%%%%%%%%%%%%%%%%%%
\usepackage[hyphens]{url}

  \journal{PubliER2022 - The industry of the future : beyond the
Anthropocene. 2-4 Feb 2022 Troyes (France)} % Sets Journal name


\usepackage{lineno} % add
\providecommand{\tightlist}{%
  \setlength{\itemsep}{0pt}\setlength{\parskip}{0pt}}

\usepackage{graphicx}
%%%%%%%%%%%%%%%% end my additions to header

\usepackage[T1]{fontenc}
\usepackage{lmodern}
\usepackage{amssymb,amsmath}
\usepackage{ifxetex,ifluatex}
\usepackage{fixltx2e} % provides \textsubscript
% use upquote if available, for straight quotes in verbatim environments
\IfFileExists{upquote.sty}{\usepackage{upquote}}{}
\ifnum 0\ifxetex 1\fi\ifluatex 1\fi=0 % if pdftex
  \usepackage[utf8]{inputenc}
\else % if luatex or xelatex
  \usepackage{fontspec}
  \ifxetex
    \usepackage{xltxtra,xunicode}
  \fi
  \defaultfontfeatures{Mapping=tex-text,Scale=MatchLowercase}
  \newcommand{\euro}{€}
\fi
% use microtype if available
\IfFileExists{microtype.sty}{\usepackage{microtype}}{}
\usepackage[left=3cm, right=3cm,top=3cm,bottom=2cm]{geometry}
\bibliographystyle{elsarticle-harv}
\ifxetex
  \usepackage[setpagesize=false, % page size defined by xetex
              unicode=false, % unicode breaks when used with xetex
              xetex]{hyperref}
\else
  \usepackage[unicode=true]{hyperref}
\fi
\hypersetup{breaklinks=true,
            bookmarks=true,
            pdfauthor={},
            pdftitle={Integrating and prioritizing ecosystems services at early development stages of industrial systems in a territory: the case of distributed recycling at Nancy, France},
            colorlinks=true,
            urlcolor=blue,
            linkcolor=blue,
            pdfborder={0 0 0}}
\urlstyle{same}  % don't use monospace font for urls

\setcounter{secnumdepth}{0}
% Pandoc toggle for numbering sections (defaults to be off)
\setcounter{secnumdepth}{0}

% Pandoc citation processing

% Pandoc header
\usepackage{float}
\usepackage{subfig}
\usepackage[utf8]{inputenc}
\def\tightlist{}
\usepackage[bitstream-charter]{mathdesign}
\usepackage{pdflscape}
\usepackage{svg}
\usepackage{lineno}
\usepackage{setspace}
\newcommand*{\doverline}[1]{\overline{\overline{#1}}}
\usepackage{tabu}



\begin{document}
\begin{frontmatter}

  \title{Integrating and prioritizing ecosystems services at early
development stages of industrial systems in a territory: the case of
distributed recycling at Nancy, France}
    \author[UTT]{Fabio A. Cruz\corref{1}}
   \ead{cruzsanc1@univ-lorraine.fr} 
    \author[UTT]{Nadège Troussier}
  
    \author[UL]{Hakim Boudaoud}
  
    \author[UL]{Mauricio Camargo}
  
      \address[UTT]{Universite de Technologie de Troyes, Troyes, France}
    \address[UL]{ERPI, Université de Lorraine, F-54000 Nancy, France.}
      \cortext[1]{Corresponding Author}
  
  \begin{abstract}
  Ecosystem services (ES) is a powerful conceptual framework to put in
  evidence the benefits that humans receive from nature, most of time
  for free. The problem is that it is complex to link between the local
  ecosystem services of an urban territory and the industrial systems
  placed within it, identifying the priority interactions (as synergy
  and impacts). From a decision-maker perspective, there is no a aid
  decision tool to guide a multicriteria evaluation at early development
  stages of industrial systems. On the ecological side, there have been
  major efforts in the valuation of ES given by ecosystems (terrestrial,
  aquatic, atmospheric) to the human well-being and great advances in
  the developing of a standard ES baselines such as Common International
  Classification of Ecosystem Services (CICES). While on the industrial
  side, life cycle assessment (LCA) tool is focused on the
  quantification of the environmental impact of industrial
  interventions. Despite its ability to cover mutually exclusive and
  exhaustive impact categories, the LCA approach still has deficiencies:
  1) it focuses on quantifying and reducing the net environmental
  impacts, but not on reducing ecological overshoot and establishing
  synergies with nature; 2) it considers the interactions between
  technological processes, but not the interactions between relevant
  ecosystems. Few researches have addressed the alignment the
  territorial priorities in terms of ES for planning and urban
  development with the supply/demand of ES by industrial systems. There
  is a need to include the ES in the desicion making process to
  establish the capacity of the territory in order to evaluate an
  absolute environmental sustainability of industrial systems. The
  purpose of this article is to propose a methodological approach in
  order to include ecosystem services in regarding the territorial and
  industrial endeavors. A prioritization is made based on the connection
  of urban ES services and the techno-ecological synergy. The results of
  are step forwards to create techno-ecological synergies between
  ecological and industrial systems. This methodological steps will
  applied to the case of distributed recycling via additive
  manufacturing (DRAM) to highlight the relevant ES from the CICES
  framework for the territory of Nancy, France. The technical
  advancements of recycling approaches using additive manufacturing are
  promising technical interventions to foster plastic recycling at a
  local level.
  \end{abstract}
  
 \end{frontmatter}

\linenumbers


\end{document}
